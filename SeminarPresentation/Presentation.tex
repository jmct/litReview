% !TEX TS-program = pdflatex
% !TEX encoding = UTF-8 Unicode

% This file is a template using the "beamer" package to create slides for a talk or presentation
% - Giving a talk on some subject.
% - The talk is between 15min and 45min long.
% - Style is ornate.

% MODIFIED by Jonathan Kew, 2008-07-06
% The header comments and encoding in this file were modified for inclusion with TeXworks.
% The content is otherwise unchanged from the original distributed with the beamer package.

\documentclass{beamer}


% Copyright 2004 by Till Tantau <tantau@users.sourceforge.net>.
%
% In principle, this file can be redistributed and/or modified under
% the terms of the GNU Public License, version 2.
%
% However, this file is supposed to be a template to be modified
% for your own needs. For this reason, if you use this file as a
% template and not specifically distribute it as part of a another
% package/program, I grant the extra permission to freely copy and
% modify this file as you see fit and even to delete this copyright
% notice. 


\mode<presentation>
{
  \usetheme{default}
  % or ...

  \setbeamercovered{transparent}
  % or whatever (possibly just delete it)
}


\usepackage[english]{babel}
% or whatever

\usepackage[utf8]{inputenc}
% or whatever

\usepackage{times}
\usepackage[T1]{fontenc}
\RequirePackage[numbers,sort&compress,square,comma]{natbib}
\bibliographystyle{IEEEtranN}
% Or whatever. Note that the encoding and the font should match. If T1
% does not look nice, try deleting the line with the fontenc.


\title[Parallel Graph Reduction] % (optional, use only with long paper titles)
{Easy parallel functional programming, the hard way.}

\subtitle
{} % (optional)

\author[] % (optional, use only with lots of authors)
{J. M. Calderon Trilla}
% - Use the \inst{?} command only if the authors have different
%   affiliation.

\institute[University of York] % (optional, but mostly needed)
{%
  Department of Computer Science\\
  PLASMA Research Group \\
  University of York
 }
% - Use the \inst command only if there are several affiliations.
% - Keep it simple, no one is interested in your street address.

\date[] % (optional)
{19-01-2012 / Literature Review Seminar}

\subject{Talks}
% This is only inserted into the PDF information catalog. Can be left
% out. 



% If you have a file called "university-logo-filename.xxx", where xxx
% is a graphic format that can be processed by latex or pdflatex,
% resp., then you can add a logo as follows:

% \pgfdeclareimage[height=0.5cm]{university-logo}{university-logo-filename}
% \logo{\pgfuseimage{university-logo}}



% Delete this, if you do not want the table of contents to pop up at
% the beginning of each subsection:
\AtBeginSubsection[]
{
  \begin{frame}<beamer>{Outline}
    \tableofcontents[currentsection,currentsubsection]
  \end{frame}
}


% If you wish to uncover everything in a step-wise fashion, uncomment
% the following command: 

%\beamerdefaultoverlayspecification{<+->}


\begin{document}

\begin{frame}
  \titlepage
\end{frame}

\begin{frame}{Outline}
  \tableofcontents
  % You might wish to add the option [pausesections]
\end{frame}


% Since this a solution template for a generic talk, very little can
% be said about how it should be structured. However, the talk length
% of between 15min and 45min and the theme suggest that you stick to
% the following rules:  

% - Exactly two or three sections (other than the summary).
% - At *most* three subsections per section.
% - Talk about 30s to 2min per frame. So there should be between about
%   15 and 30 frames, all told.

\section{Introduction}

\subsection[Motivation]{Motivation}

\begin{frame}{Why Functional Programming?}
    \begin{itemize}
        \item
            The lack of state in pure functional languages makes parallel programming easier to 
            reason about. 
        \item
            The largest sucesses in large scale parallelism have come from the declarative
            and functional paradigms (Databases, Map/Reduce).
    \end{itemize}
\end{frame}

\begin{frame}{Why Not Imperative Programming?}
    \begin{itemize}
        \item
            Mutable state introduces more resource sharing complications that can lead to dealock
        \item
            Because of the `step-by-step' nature of imperative programming, new language 
            constructs are needed to support parallelism. They will not work ``as is''. 
    \end{itemize}
\end{frame}
\begin{frame}[fragile]{Understanding Potential Wells}{Potential Wells}

\end{frame}

\begin{frame}[fragile]{What GRIP looks like}{}

\begin{figure}[h]
 \centering
 \includegraphics[scale=.4]{figures/GRIP.png}
 \caption{A circuit board for the GRIP machine in the making \cite{PFPAnIntro}}
\end{figure}
\end{frame}


\begin{frame}{Quantum Dots as Artificial Atoms}{}
  % - A title should summarize the slide in an understandable fashion
  %   for anyone how does not follow everything on the slide itself.

  Quantum Confinement in 3D -> Artificial atom
  \begin{itemize}
  \item
   Question: What atomic properties are we looking to exploit?
   \pause
  \item
    Answer: Fluorescence.
  \end{itemize}
\end{frame}


\section*{Questions}

\begin{frame}
\centering
Questions
\end{frame}
\bibliography{seminar}

\end{document}


