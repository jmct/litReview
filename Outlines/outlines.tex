\documentclass[11pt]{article}

\usepackage{outlines}
\usepackage{fancyvrb}
\DefineShortVerb{\"}

\title{The \texttt{outlines} package}
\author{Charles Pecheur}
\date{March 25, 2005}

\begin{document}

\maketitle

\begin{abstract}
The \texttt{outlines} package defines the \texttt{outline} environment,
that allows outline-style indented lists with freely mixed levels up
to four levels deep.  It replaces the nested "begin"/"end" pairs by
different item tags "\1" to "\4" for each nesting level.  This is very
convenient in cases where nested lists are used a lot, such as for to-do
lists or presentation slides.
\end{abstract}

\section{Example}
This is an example of basic usage:


{\small\begin{verbatim}
\begin{outline}
  \1 This is a first item.
  \1[!!!] This is a second, with a custom label.
    \2 A level-2 item.
      \3 A level 3.
        \4 Deepest is level 4.
    \2 Back to level 2.
\0 A normal paragraph in the middle.
  \1 A couple more  
    \2 items.
\end{outline}
\end{verbatim}}

Produces:

{\small
\begin{outline}
  \1 This is a first item.  
  \1[!!!] This is a second, with a custom label.
    \2 A level-2 item.
      \3 A level 3.
        \4 Deepest is level 4.
    \2 Back to level 2.
\0 A normal paragraph in the middle.
  \1 A couple more
    \2 items.
\end{outline}
}

\section{Usage}

\begin{outline}

\0 In the preamble:

\1 "\usepackage{outlines}" 

loads this package (no options supported).

\0 In the document:

\1 "\begin{outline}["\emph{style}"]" \emph{body} "\end{outline}" 

produces an \emph{outline} region, with a hierarchy of items up to four levels deep.  The outline is formatted according to \emph{style}, which must be the name of a \LaTeX\ list environment.  The default is "itemize".  All levels use the same style.

\0 Inside \emph{body}:

\1 "\1["\emph{lbl}"]", "\2["\emph{lbl}"]", "\3["\emph{lbl}"]",
"\4["\emph{lbl}"]"

introduce outline items at the four nesting levels.  They are
used the same way as "\item["\emph{lbl}"]" in list environments, where
\emph{lbl} is an optional custom item label.  

\1 "\0"

introduces a normal, non-itemized paragraph.

\end{outline}

\section{Limitations}

\LaTeX\ list environments cannot begin with a nested list.
In outlines, that means that a level-$n$ item may only follow an item
of level $n-1$ or higher.  For example, the following produces two
``"missing \item"'' errors:

\begin{verbatim}
\begin{outline}
    \2 Missing level 1,
        \4 missing level 3.
\end{outline}
\end{verbatim}

Do not use outlines inside other outlines or other list environments.
Nested lists in outlines should work and be consistent with the current
level of the outline (e.g. a nested list following a level-2 outline
item will look as a level-3 list). The four-level limit applies overall.

\section{Implementation Notes}

The package is implemented in \LaTeX\ (no plain \TeX); it should be
easy to understand and customize even to a non-\TeX-pert.  The main
programming trick is a set of commands "\ol@toz", \dots, "\ol@toiiii"
which are dynamically modiified to contain the necessary list openings or closings to reach outline levels 0 to 4 from the current level.

Outlines expand to the corresponding hierarchy of nested lists of the
selected style.  All custom list formatting and user-provided list styles
should be compatible with "outline" environments, as long as they keep
the "\item" syntax.

\section{Credits}

This package was developed by Charles Pecheur at Universit\'e catholique de
Louvain, Belgium.  It is free for anyone to use, modify and re-distribute
as long as credit to the original author is preserved.  Charles Pecheur
can be contacted at "pecheur@info.ucl.ac.be".

This package is independent from similar packages "outline.sty" and "outliner.sty", available on the CTAN archive.

\end{document}   
